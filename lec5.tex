\section{Lektion 5, övningsuppgifter}

\subsection{Uppgift 1}
\textit{Låt \(x_1,x_2,\ldots,x_n\) vara en sekvens av distinkta heltal. Vi säger
att denna sekvens är cykliskt sorterad om det minsta talet är $x_i$ (för något
okänt $i$) och \(x_i,x_{i+1},\ldots,x_n,x_1,\ldots,x_{i-1}\) är sorterad i
stigande ordning. Utveckla en algoritm för att bestämma positionen av det minsta
talet i en cykliskt sorterad sekvens. Algoritmen måste löpa snabbare än i
\(O(n)\) tid.}



\subsection{Uppgift 2}
\textit{}