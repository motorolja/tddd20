\section{Lektion 1}
\(sum_i i = 1 \rightarrow n = n(n+1)/(2)\)
Induktion:
Bassteg: testa n=1

Induktionshypotesen: Antag att likheten gäller för godtyckligt n, e.g. n = k

Induktion steg: Visa att likheten gäller då n = k+1
sumi i=1 -> k+1 = ()

Motsägelsebevis:
För att visa påstående P antag icke P. Visa att det leder till motsägelse.

x^2 - y^2 = 1  saknar heltalslösningar då x,y>0
Antag att x,y>0 är heltal och x^2 - y^2 = 1
konjugat: (x+y)(x-y) = 1
I: x+y = 1, x-y = 1 -> x = 1, y = 0
II: x+y=-1, x-y=-1 -> x = -1, y = 0

Algoritm analys:
tre metoder för att ränka ordo(kan ha missförståt), en för söndra härska, en rekursiva

Fibonachi talen
T(1)=T(2)=1
T(n)=T(n-1) + T(n-2)

svar: 'O(F(n))' O(c^2) där c är en konstant

indata --> A --> svar

A går i polynomisk tid omm
\begin{itemize}
  A tar max p(||indata||) tid där p är ett fixt polynom.
\end{itemize}

En proposition är en variabel som kan ta två värden {0,1} (true/false)
och \(\and\) ^
eller \(\or\) v
icke \(not\)

CNF = Conjugtive Normal Form
\begin{equation}
  (p \or q \or r) \and (p \or r) \and (r \or q)
\end{equation}

En graf: (V,E) där V är en mängd hörn och E en mängd kanter

En nods omgivning: Angränsande noder samt kanske sig själv beroende på vem man
frågar

Klick: Kanter mellan alla i mängden
Oberoende mängd: inte Klick

