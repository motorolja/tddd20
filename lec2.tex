\section{Lecture 2}
\begin{itemize}
\item{\textbf{Giriga algoritmer:}
    \textit{2 - färgning:}
    
    indata: En graf \(G=(V,E)\)
    Fråga: Kan grafens noder färgas med två färger?

    En graf med 4 samanbundna noder, varje nod med gradtal 2. Svar ja.

    Motivering:

    \begin{array}{l | l}
      JA & NEJ \\\hline
      Ja man kan lita på p.g.a. att konkret lösning, konstrueras.

         & Udda antal noder kan alltid med graphanalys där antal noder är 1
           eller \(\g 3\) alltid färgas. Bevis antag att detta inte är fallet.
                                                                
    \end{array}
    \textit{Nodhölje(täckning av alla kanter från x antal noder):}

    Täckning med minimalt antal noder:
    \begin{enumerate}
    \item Börja vid ett löv
    \item Gå en nivån up till nästa nod via en kant
    \item Ta bort alla kanter som är täckta av noden
    \item Repitera tills alla kanter är täckta
    \end{enumerate}

    \textit{Punkt-intervall/täckning:}
    \begin{itemize}
    \item{Indata: Mängd punkter, mängd intervall}
    \item{Utdata: En delmängd av intervallen som täcker alla punkter och har
        minimal storlek.}
      \item{Bevis: Antag att \(A\) inte är en optimal algoritm för problemet. Då
          finns en instans punkt-intervall \(I\) så att
          \(|(OPT(I)| \lt |(A(I))|\)
          Antag att \(I\) har minimalt antal punkter. \(OPT(I)\) täcker \(x\)
          med \(J_B^1,\ldots,J_B^k\) svaret från \(A\) täcker x med \(J_A\)
          Betrakta instansen \(I\prime\) som är \(I\) där punktmängden
          reducerats med alla punkter som täcks av \(J_A\).
          \begin{equation}
            |A(I\prime)| = |A(I)| - 1
            |OPT(I\prime)| \lt |OPT(I)| \lt |A(I)| = |A(I\prime)| + 1
            |A(I)| - |OPT(I\prime)| \gt 2
            |A(I\prime)| + 1 - |OPT(I\prime)| \gt 2
            [I\prime innehåller färre punkter än I |A(I\prime)| = |OPT(I\prime)|]
            |A(I\prime)| + 1 - |A(I\prime)| \gt 2
            0 \gt 1
          \end{equation}
      }
    \end{itemize}
      
    
  }
\item{\textbf{Rekursiv nedbrytning:}

  }
\item{\textbf{Dynamisk programmering:}
    
  }
\end{itemize}
