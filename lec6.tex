\section{Lektion 6}

\subsection{Beräkningskomplexitet, I}
Skillnaden mellan algoritmkomplexitet och problemkomplexitet.\\

\begin{itemize}
\item{Algoritmkomplexitet:
    Det vi arbetat med innan.
  }
\item{Problemkomplexitet:
    Behandlar mer flummiga betraktelser. \\
    \begin{itemize}
      \item{Problem: Ett problem är en funktion som tar en indata mängd och ger
          utdata i form av ett svar. Finns olika typer, lösbara(beräkningsbara),
          icke beräkningsbara.}
      \item{beräkningsbara problem: Ska finnas en algoritm som beräknar svaret.
          Den ska köra i ändlig tid, ej ta oändliga mängder (uppräkningsbar) som
          indata. }
      \item{NP
          Problemet \(x\epsilon P\) om det kan lösas i polynomisk tid \(X
          \epsilon NP\) om lösningar kan verifieras i polynomisk tid.
        }
    \end{itemize}
  }
\end{itemize}

\begin{equation}
  P = NP
  \text{Indata: } \oslash
  \text{Fråga: Är } P = NP \text{?}
\end{equation}

\subsubsection{Polynomisk reduktion}
Om du har ett svårt problem \(P_a\) och vill vissa att ett godtyckligt problem
\(P_i\), givet en polynomisk algoritm $A$, kan du visa att \(P_i\) är svårt?
\begin{itemize}
\item{\(P_a\) ska generera samma svar efter $A$.}
\item{Vad är ett ``svåraste'' problem i \(NP\)?\\
    
  }
\item{Om \(X\) är \(NP\) fullständig och \(X \epsilon P\) är alla problem i
    \(P\) fullständiga.}
\end{itemize}

\paragraph{Sidnoter}
Se ordet reduktion i ``Polynomisk reduktion'' bara som ett ord.
