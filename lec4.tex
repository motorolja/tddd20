\section{Lektion 4}
\textbf{Dynamisk Programmering}

\begin{itemize}
\item{\textbf{Idata:}

    En mängd positiva heltal $I$ s.a. \(1 \epsilon I\), ett positivt heltal $x$
  }
\item{\textbf{Utdata:}
    En sekvens \(S = (i_1,\ldots,i_k)\) av tal från $I$ s.a. \(x =
    \sum{i_j}{j=1}{k}\) och $S$ är så kort som möjligt.
  }

\item{\textbf{Lednoter för lösning:}
    Matrislösning(\( 1 = man nått fram, 0 = ej nått fram\))
    \begin{itemize}
      \item $j$ på ena axeln, där $j$ indikerar sekvensens längd
      \item $i$ mostsvarar det tal man nått fram till,
      \end{itemize}
      Induktions bevis är lämpligt för denna uppgift, varför?\\

      Tidskomplexitet, mätt i indata storlek: \(O((2^{||x||})^2*|I|)\)\\
      \(\exp(a,n)\)
      \begin{itemize}
      \item om $n$ är jämn: \(c=\exp(a,n/2)\) ret \(c*c\) ger \(O(\log{2}*n)\)
      \item om $n$ är udda: \(c=\exp(a,\div{n-1}{2})\) ret \(a*c*c\) ger \(O(2^{\log{n}*n})
        = O(||n||)\)

      \end{itemize}
  }
  
\end{itemize}
  

\subsection{Sidnoter}

Vilken metod ska man använda? (dynamisk programmering, rekursiv nedbrytning)\\
Svar: Gå på känsla