\section{Lektion 7}

Denna lektion handlar om reduktioner.

\subsection{Reduktion}
\begin{itemize}
\item{Nytta, ha ett problem som är svårt och visa att reduktioner är svåra.}
\item{Resolution: inte klar!\\
    \(om (p\and A) (\not p\and B)\)}
\item{Nej är svårt}
\end{itemize}

\paragraph{Exempel 1}
\begin{equation}
  SAT \rightarrow 3SAT

\end{equation}

\paragraph{Exempel 2}
Det finns en tilldelning till dessa som satisfieras. Gå ifrån reduktionen till
orginal problemet. 
\begin{equation}
  O((n*\log{n}) * m)
\end{equation}

\paragraph{Exempel 3}

\(3SAT \rightarrow \text{Obegränsad mängd}\)

\begin{itemize}
\item{Instans: Graf \(G = (V,E)\) och ett heltal \(K \geq O\)}
\item{Fråga: Innehåller \(G\) en oberoende mängd med minst \(K\) noder?
 }
\item{Svar: Ja \(\rightarrow\) Ja \\
  \((x_1 \and x_2 \and x_3) \or (\not x_1 \and \not x_2 \and \not x_3) \or (\not
   x_1 \and x_2 \and x_3)\) 
  
  }
\end{itemize}

\paragraph{Nodhölje}
\begin{itemize}
\item{Instans: Graf \(G = (V,E)\),  heltal \(K \geq 1\)}
\item{ Finns nodhölje av storlek \(K\) eller mindre?

  }
\item{Svar: Reduktion från oberoende mängd. \([G = (V,E), K]\) \\
    \([G, |V| - K]\)

  }
\end{itemize}